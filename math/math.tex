\documentclass[12pt, a4paper]
{article}

\usepackage{svg}
\setsvg{inkscape=inkscape -z -D,svgpath=images/}
\usepackage{float}
\usepackage{amsmath}
\usepackage{todonotes}

\providecommand{\todos}[1]{\todo[inline]{#1}}
\providecommand{\ang}[1]{\theta_{\text{#1}}}
\providecommand{\sub}[1]{_{\text{#1}}}
\providecommand{\subi}[1]{_{\text{#1},i}}
\providecommand{\twonorm}[1]{\left|\left|#1\right|\right|}
\providecommand{\foralli}{\quad \forall\, i}
\providecommand{\lr}[1]{\left( {#1} \right)}

\providecommand{\tp}{\tilde{p}}
\providecommand{\inert}{w}
\providecommand{\bp}{\boldsymbol{p}}
\providecommand{\btp}{\tilde{\bp}}

\providecommand{\xh}{\hat{x}}
\providecommand{\yh}{\hat{y}}

\providecommand{\qcom}{, \qquad}


\title{Reference frame conversions}
\author{Laurens Valk}
\begin{document}
\maketitle

\tableofcontents

\section{Preliminaries}

\subsection{Points and Vectors}
\begin{equation}
r_{b/a}^c = p^c_b - p^c_a
\end{equation}
\todos{add picture}
\subsection{Transformation}

\begin{equation}
p^i = \alpha\lr{R^i_j p^j + r^i_{j/i}}
\end{equation}

\begin{equation}
\begin{bmatrix}
p^i\\
1
\end{bmatrix} = \underbrace{\begin{bmatrix}
R^i_j & r^i_{j/i}\\
0 & 1
\end{bmatrix}}_{H^i_j}\begin{bmatrix}
p^j\\1
\end{bmatrix}
\end{equation}

\subsection{Inverse Transformation}

\begin{equation}
\begin{bmatrix}
p^j\\
1
\end{bmatrix} = \underbrace{\begin{bmatrix}
(R^i_j)^T & -(R^i_j)^Tr^i_{j/i}\\
0 & 1
\end{bmatrix}}_{H^i_j}\begin{bmatrix}
p^j\\1
\end{bmatrix}
\end{equation}

\section{From Pixels to Meters} 
\subsection{Transformation Matrices}
The symbol $\tp$ expresses points in units of pixels. Any other point $p$ has position units of meters.
\begin{figure}[H]
\centering
\includesvg[width=1\textwidth]{scaling}
\caption{Four coordinates: a) in the camera picture, b) with flipped y axis, c) with the origin at the midbase of the picture, d) scaled to physical distances.
\label{fig:robotframe}}
\end{figure}

\begin{equation}
\begin{bmatrix}
\tp^f\\
1
\end{bmatrix} = \begin{bmatrix}
1 & 0 & 0\\
0 & -1 & 0\\
0& 0 & 1
\end{bmatrix}\begin{bmatrix}
\tp^p\\1
\end{bmatrix}=H^f_p \begin{bmatrix}
\tp^p\\1
\end{bmatrix}
\end{equation}
\begin{equation}
\begin{bmatrix}
\tp^c\\
1
\end{bmatrix} = \begin{bmatrix}
1 & 0 & -\frac{1}{2}\tilde{d}_x\\
0 & 1 & \frac{1}{2}\tilde{d}_y\\
0& 0 & 1
\end{bmatrix}\begin{bmatrix}
\tp^f\\1
\end{bmatrix}=H^c_f \begin{bmatrix}
\tp^f\\1
\end{bmatrix}
\end{equation}
\begin{equation}
\begin{bmatrix}
p^{\inert}\\
1
\end{bmatrix} = \begin{bmatrix}
\varepsilon & 0 & 0\\
0 & \varepsilon & 0\\
0& 0 & 1
\end{bmatrix}\begin{bmatrix}
\tp^c\\1
\end{bmatrix}=H^{\inert}_c \begin{bmatrix}
\tp^c\\1
\end{bmatrix}
\end{equation}

\begin{equation}
\begin{bmatrix}
p^{\inert}\\
1
\end{bmatrix} = H^{\inert}_c H^c_f H^f_p \begin{bmatrix}
\tp^p\\1
\end{bmatrix}
\end{equation}
%
The overall transformation from pixels to the real world representation is
%
\begin{equation}
\begin{bmatrix}
p^{\inert}\\
1
\end{bmatrix} = H^{\inert}_p \begin{bmatrix}
\tp^p\\1
\end{bmatrix} \quad \Rightarrow \quad \bp^{\inert} = H^{\inert}_p 
\btp^p
\end{equation}


\todos{fix scaling outside $H$}

\subsection{Scaling Factor}
%
\todos{consider working with centimeters}
The distance $d\sub{apex/midbase}$ in Figure \ref{fig:robotframe} is known, constant, and equivalent for each agent:
%
\begin{equation}
d\sub{apex/midbase} \equiv \twonorm{p\subi{apex}-p\subi{midbase}} \foralli
\end{equation}
%
Per agent, we can estimate the scaling factor $\epsilon_i$ as:
%
\begin{equation}
\varepsilon_i = \dfrac{d\sub{apex/midbase}}{\twonorm{\tp\subi{apex}-\tp\subi{midbase}}} \quad \text{m/pixel}
\end{equation}
%
A more accurate estimate for the overall scaling factor is:
%
\begin{equation}
\varepsilon = \dfrac{1}{N}\sum^N_{i=1} \varepsilon_i
\end{equation}
%
However, because $\varepsilon$ is constant it may be more desirable to calculate it once and use it as a constant parameter. If so, we may divide the arena width by the picture with in pixels:
%
\begin{equation}
\varepsilon = \dfrac{d\sub{field}}{\tilde{d}_x} \quad \text{m/pixel}
\end{equation}
\pagebreak
\section{Local Robot Frames}
\todos{cross check p's and then define vector as difference between two points; and then vectors cannot be transformed; just rotated}

\subsection{From Label Markers to Label Frame}

\begin{figure}[H]
\centering
\includesvg[width=0.5\textwidth]{labelworld}
\caption{Local vehicle frame definitions
\label{fig:labelworld}}
\end{figure}


First, transform pixel locations to world coordinates:
%
\begin{equation}
\bp\sub{midbase}^w = H^{\inert}_p \btp\sub{midbase}^p, \quad \bp\sub{apex}^w = H^{\inert}_p \btp\sub{apex}^p
\end{equation}
%
Obtain the label x and y axes in world coordinates:
%
\begin{align}
\xh_\ell^{\inert}= \dfrac{p\sub{apex}^w-p\sub{midbase}^w}{\twonorm{p\sub{apex}^w-p\sub{midbase}^w}} \qcom \yh_\ell^{\inert} = \begin{bmatrix}
0 & -1\\
1 & 0
\end{bmatrix}\xh_\ell
\end{align}
%
The transformation matrix becomes:
%
\begin{equation}
\bp^{\inert} =H^{\inert}_\ell 
\bp^\ell\qcom H^{\inert}_\ell = \begin{bmatrix}
\xh_\ell^{\inert} & \yh_\ell^{\inert} & r^{\inert}\sub{midbase/origin}\\
0 & 0 & 1
\end{bmatrix}
\end{equation}


\todos{Agree on a better name than center. This is confusing}
\todos{replace terminology: top = apex, center = midbase}
\todos{FIX: Inside H matrix are always VECTORS. not p's.}
\subsection{From Label Frame to Robot Frame}

\todos{distinguish between points and vectors: Then vectors suddenly make sense: points in a frame are always defined with respect to the origin of that frame}
\begin{equation}
\bp^b = \begin{bmatrix}
I_2 & r^b\sub{midbase/robot}\\
0& 1
\end{bmatrix}\bp^\ell
\end{equation}

\begin{figure}[H]
\centering
\includesvg[width=0.7\textwidth]{robotframe}
\caption{Local vehicle frame definitions
\label{fig:robotframe}}
\end{figure}

\section{Relative Agent Positions and Orientations}




\end{document}